\documentclass[mathserif]{beamer}
\usepackage{indentfirst}
\usepackage{amsmath,amsthm,amssymb}
\usepackage{fancybox}
\usepackage{fancyvrb}
%\usepackage{minted}
\usepackage{color}
\usepackage{makeidx}
\usepackage{xeCJK}
%\setCJKmainfont[BoldFont={Adobe Heiti Std}, ItalicFont={AR PL New Kai}]{Adobe Song Std}
\setCJKmainfont[BoldFont={Adobe Heiti Std}, ItalicFont={Adobe Kaiti Std}]{Adobe Song Std}
\usepackage{graphicx}
\usepackage{geometry}
\usepackage{amsmath}
\usepackage{amsfonts}
\usepackage{array}
\usepackage{gnuplot-lua-tikz}
\usepackage{cite}
\usepackage{url}
\usepackage{hyperref}
\hypersetup{colorlinks=true,linkcolor=blue,anchorcolor=green,citecolor=blue}
\usepackage{wrapfig}
%\usepackage{lettrine}
%\usepackage{abstract}

% THEOREMS -------------------------------------------------------
\newtheorem{Thm}{Theorem}
\newtheorem{Cor}[Thm]{Corollary}
\newtheorem{Conj}[Thm]{Conjecture}
\newtheorem{Lem}[Thm]{Lemma}
\newtheorem{Prop}[Thm]{Proposition}
\newtheorem{Prob}{Problem}
\newtheorem{Exam}{Example}
\newtheorem{Def}[Thm]{Definition}
\newtheorem{Rem}[Thm]{Remark}
\newtheorem{Not}[Thm]{Notation}
\newtheorem*{Sol}{Solution}

% MATH -----------------------------------------------------------
\newcommand{\norm}[1]{\left\Vert#1\right\Vert}
\newcommand{\abs}[1]{\left\vert#1\right\vert}
\newcommand{\set}[1]{\left\{#1\right\}}
\newcommand{\Real}{\mathbb R}
\newcommand{\eps}{\varepsilon}
\newcommand{\To}{\longrightarrow}
\newcommand{\BX}{\mathbf{B}(X)}
\newcommand{\A}{\mathcal{A}}
\newcommand{\CommentS}[1]{}
% CODE ----------------------------------------------------------
\newcommand{\PltImg}[1]{
\begin{center}
\input{#1}
\end{center}
}

\newenvironment{code}%
{\vglue 5pt \VerbatimEnvironment\begin{Sbox}\begin{minipage}{0.9\textwidth}\begin{small}\begin{Verbatim}}%
{\end{Verbatim}\end{small}\end{minipage}\end{Sbox}\setlength{\shadowsize}{2pt}\shadowbox{\TheSbox}\vglue 5pt}


\usepackage{pgf}
%\usepackage{tikz}
%\usetikzlibrary{arrows,automata}
%\usepackage[latin1]{inputenc}
\usepackage{verbatim}
\usepackage{listings}
%\usepackage{algorithmic} %old version; we can use algorithmicx instead
\usepackage{algorithm}
\usepackage[noend]{algpseudocode} %for pseudo code, include algorithmicsx automatically

\lstdefinelanguage{Smalltalk}{
  morekeywords={self,super,true,false,nil,thisContext}, % This is overkill
  morestring=[d]',
  morecomment=[s]{"}{"},
  alsoletter={\#:},
  escapechar={!},
  literate=
    {BANG}{!}1
    {UNDERSCORE}{\_}1
    {\\st}{Smalltalk}9 % convenience -- in case \st occurs in code
    % {'}{{\textquotesingle}}1 % replaced by upquote=true in \lstset
    {_}{{$\leftarrow$}}1
    {>>>}{{\sep}}1
    {^}{{$\uparrow$}}1
    {~}{{$\sim$}}1
    {-}{{\sf -\hspace{-0.13em}-}}1  % the goal is to make - the same width as +
    %{+}{\raisebox{0.08ex}{+}}1		% and to raise + off the baseline to match -
    {-->}{{\quad$\longrightarrow$\quad}}3
	, % Don't forget the comma at the end!
  tabsize=2
}[keywords,comments,strings]

\lstloadlanguages{C++, Lisp, Haskell, Python, Smalltalk, Mathematica, Java,bash,Gnuplot,make,Matlab,PHP,Prolog,R,Ruby,sh,SQL,TeX,XML}


\title{What Makes Fatworm in Java a Terrible Idea}
\author{Jingcheng Liu}
\institute{ACM 2010}
%\usetheme{Warsaw}
\usetheme{Madrid}
%\usetheme[numbers,totalnumber,compress,sidebarshades]{PaloAlto}
%\usetheme{Copenhagen}
%\usecolortheme{whale}
\usecolortheme{seahorse}
%\setbeamercolor{background canvas}{bg=blue!9} 
\setbeamertemplate{theorems}[ams style]
\setbeamertemplate{navigation symbols}{}      

\begin{document}

\frame{\titlepage}
\frame{
	\frametitle{Outline}
	\tableofcontents
}
\section{Motivating Example}
\frame{
	\frametitle{Examples}
Here I'm going to demonstrate how a simple change of coding style leads to 16x performance gain,
which could easily overwhelm many more general optimizations on Logic Plan.
\begin{Exam}
	Evaluating binary expressions.%: sort/sort, indexTest/testIndex_hugeIndex.
\end{Exam}
%should I push Select through Rename?
%precise env construction
%rename expressions with a shorter name
%fromString, toString, fromDecimal, toDecimal
%hugeIndex : precise interval => mark un-necessary predicate evaluation.
\begin{Exam}
	Product list faster than binary product?
\end{Exam}
%after equivalent expressions and constant propogation, and manual direction of adjusting product plan, no significant difference given these data.
%what really make a difference is Java method call => no tail call optimization, no continuation passing style
}
\section{Analysis}
\frame{
	\frametitle{What leads to such huge difference}
	Besides alias, the source of all other evils: Java and JVM.

	\begin{itemize}
			\item Object is slow
			\item Method call/recursion is slow(no tail call optimization)
			\item Speed of evaluating binary expressions(wraping and unwraping of String/Decimal)
	\end{itemize}
}

\section{Conclusion}
\frame{
	\frametitle{Conclusion}
\begin{itemize}
	\item Had there been no alias, there would be no more wars.
	\item Java didn't guarantee a natural but efficient way to write the code. % checkout Haskell's Monad!
	\item If you have to tackle the details yourselves why bother Java, we already have C/C++.
\end{itemize}
}
\end{document}
